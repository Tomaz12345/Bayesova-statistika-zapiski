\documentclass[a4paper, 12pt]{book}

\usepackage{fancyhdr}

\newcommand{\ttitle}{Bayesova statistika - zapiski s predavanj prof. Smrekarja}
\newcommand{\ttitleshort}{Bayesova statistika}
\newcommand{\tauthor}{Tomaž Poljanšek}
\newcommand{\tdate}{študijsko leto 2023/24}

\usepackage{color}
\usepackage{soul}
\usepackage[numbers]{natbib}

\usepackage{physics}

\usepackage[parfill]{parskip}
\usepackage[hyphens]{url}

\usepackage[usestackEOL]{stackengine}[2013-10-15] % formatting Pascal
\usepackage[dvipsnames]{xcolor}

\usepackage{cancel}
\usepackage[export]{adjustbox}

% Related to math
\usepackage{amsmath,amssymb,amsfonts,amsthm}
\usepackage{mathtools}
\usepackage{youngtab}
\usepackage{tikz}

% encoding and language
\usepackage{lmodern}
\usepackage[slovene, english]{babel}
\usepackage[utf8]{inputenc}
\usepackage[T1]{fontenc}

% multiline comments
\usepackage{comment}
\usepackage{verbatim}

% random text - for texting
\usepackage{lipsum}
\usepackage{blindtext}

\usepackage{hyperref}

% images
\usepackage{graphicx}
\graphicspath{ {../images/} }

% no blank page
\usepackage{atbegshi}
\renewcommand{\cleardoublepage}{\clearpage}

% theorems
\theoremstyle{definition}
\newtheorem{counter}{Counter}[section]
\newtheorem{defn}[counter]{Definicija}
\newtheorem{lemma}[counter]{Lema}
\newtheorem{conseq}[counter]{Posledica}
\newtheorem{claim}[counter]{Trditev}
\newtheorem{theorem}[counter]{Izrek}
\newtheorem{pro}[counter]{Dokaz}
%%
\theoremstyle{remark}
\newtheorem*{ex}{Primer}
\newtheorem*{exmp}{Zgled}
\newtheorem*{rem}{Opomba}

% QED
\renewcommand\qedsymbol{$\blacksquare$}

\hypersetup{pdftitle={\ttitle}}

\addtolength{\marginparwidth}{-20pt}
\addtolength{\oddsidemargin}{40pt}
\addtolength{\evensidemargin}{-40pt}

\renewcommand{\baselinestretch}{1.3}
\setlength{\headheight}{15pt}
\renewcommand{\chaptermark}[1]
{\markboth{\MakeUppercase{\thechapter.\ #1}}{}} \renewcommand{\sectionmark}[1]
{\markright{\MakeUppercase{\thesection.\ #1}}} \renewcommand{\headrulewidth}{0.5pt} \renewcommand{\footrulewidth}{0pt}

% header
\fancyhf{}
\fancyhead[LE,RO]{\sl \thepage} 
\fancyhead[RE]{\sc \tauthor}
\fancyhead[LO]{\sc \ttitleshort}


\newcommand{\autfont}{\Large}
\newcommand{\titfont}{\LARGE\bf}
\newcommand{\clearemptydoublepage}{\newpage{\pagestyle{empty}\cleardoublepage}}
\setcounter{tocdepth}{1}

\newcommand{\N}{\mathbb{N}}
\newcommand{\Z}{\mathbb{Z}}
\newcommand{\Q}{\mathbb{Q}}
\newcommand{\R}{\mathbb{R}}
\newcommand{\C}{\mathbb{C}}
\newcommand{\ch}{\operatorname{char}}

\DeclarePairedDelimiter\ceil{\lceil}{\rceil}
\DeclarePairedDelimiter\floor{\lfloor}{\rfloor}

\usepackage{float}
\usepackage{multirow}
\usepackage{icomma}
\usepackage{tabularx}
\usepackage{hhline}

\usepackage{enumitem}
\usepackage{ulem}
\newcommand{\msout}[1]{\text{\sout{\ensuremath{#1}}}} % cross text in math mode

\title{\ttitle}
\author{\tauthor}
\date{\tdate}

\newcommand\mymaketitle{
  \begin{titlepage}
    \begin{center}
        \vspace*{4cm}
        \Huge
        \textbf{\ttitle}
                        
        \vspace{1.5cm}
        \huge
        \tauthor
            
        \vspace{3cm}
        \Large
        \tdate
    \end{center}
  \end{titlepage}
}




\begin{document}

\selectlanguage{slovene}
\renewcommand{\thepage}{}
\newcommand{\sn}[1]{"`#1"'}

\mymaketitle

\clearpage
\frontmatter

% kazalo
\pagestyle{empty}
\def\thepage{}
\tableofcontents{}

%%
\def\x{\hspace{3ex}}    %BETWEEN TWO 1-DIGIT NUMBERS
\def\y{\hspace{2.45ex}}  %BETWEEN 1 AND 2 DIGIT NUMBERS
\def\z{\hspace{1.9ex}}    %BETWEEN TWO 2-DIGIT NUMBERS
\stackMath

\clearpage
\phantomsection

\section*{Seznam uporabljenih kratic}

\noindent\begin{tabular}{p{0.1\textwidth}|p{.8\textwidth}}
  {\bf kratica} & {pomen} \\
  \hline
  {\bf s.v.} & {slučajni vektor} \\
  {\bf B} & {binomska porazdelitev} \\
  {\bf NEP} & {neodvisen in enako porazdeljen} \\
  {\bf s.s.} & {slučajna spremenljivka} \\
  {\bf p.v.} & {pričakovana vrednost} \\
\end{tabular}

\clearpage
\phantomsection
\addcontentsline{toc}{chapter}{Povzetek}
\chapter*{Povzetek}


\pagenumbering{arabic}

\mainmatter
\setcounter{page}{1}
\pagestyle{fancy}


% 1. predavanje: 2.10.

\chapter{Uvod}


Bayesova statistika je formalni okvir za \sn{osveževanje} vedenja/znanja o porazdelitvi nekega slučajnega vektorja.
\begin{exmp}
  1000, $\approx 400$Č $\to 600$B (apriorno znanje). \\
  Izvedemo (statistični) poskus: izvlečemo $10$, dobimo $6$ črnih in $4$ bele
\end{exmp}


\section{Elementarna Bayesova statistika}

Privzamemo popoln sistem dogodkov $E_1, E_2 \dots E_m: E_i \cap E_j = \emptyset$ za $i \neq j$ in
$E_1 \cup E_2 \cup \dots \cup E_m = \Omega$. \\
Če imamo še neki dogodek $A$, velja t.i. zakon o popolni verjetnosti \\
$P(A) = \sum_{i=1}^{m} P(A \mid E_i) \cdot P(E_i)$ (interpretacija: 2-fazni poskus). \\
V Bayesovem okviru nas zanimajo $P(E_j \mid A)$ (verjetnost, da se je v \sn{1. fazi} zgodil $E_j$,
če se je \sn{2. fazi} zgodil $A$).
Ker je
\begin{equation*}
  P(E_j \mid A) = \frac{P(E_j \cap A)}{P(A)}
\end{equation*}
je
\begin{equation*}
  P(E_j \mid A) = \frac{P(A \mid E_j) \cdot P(E_j)}{P(A)} \quad \text{- elementarna pogojna verjetnost}
\end{equation*}
oziroma
\begin{equation*}
  P(E_j \mid A) = \frac{P(A \mid E_j) \cdot P(E_j)}{\sum_{i=1}^{m} P(A \mid E_i) \cdot P(E_i)}
    \quad \text{- elementarna Bayesova formula.}
\end{equation*}
Nadaljujemo zgled. V Bayesovi statistiki predhodno (\sn{apriorno}) vedenje formaliziramo kot realizacijo slučajnega eksperimenta.
V našem primeru vpeljemo fukcijo, da smo število črnih frnikul $\theta$ (- realizacija) dobili kot rezultat
slučajne spremenljivke $\Theta \in \{0, 1, 2 \dots 1000\}$. \\
Informacijo $\theta \approx 400$ zakodiramo kot $E(\Theta) = 400$. \\
% skica
Privzamemo (kar!) $\Theta \sim B\left(1000, \frac{4}{10}\right)$ \\
$\implies P(\Theta = \theta) = \binom{1000}{\theta} \left(\frac{4}{10}\right)^{\theta} \left(1-\frac{4}{10}\right)^{1000-\theta}$. \\
$P(k$ črnih od $10$ izvlečenih$ \mid \Theta = \theta) = \frac{\binom{\theta}{k} \binom{1000-\theta}{10-k}}{\binom{10}{k}}$ (*) \\
(*) pri omejitvah ($k$ omejimo). \\
Osvežena porazdelitev - novo vedenje \\
\begin{align*}
  &P(\Theta = \theta \mid 6 \text{ črnih od } 10 \text{ izvlečenih}) = \\
  &\frac{P(6 \text{ črnih od } 10 \text{ izvlečenih} \mid \Theta = \theta) \cdot P(B(1000, \frac{4}{10}) = \theta)}
    {\sum_{i=0}^{1000} P(6 \text{ črnih od } 10 \text{ izvlečenih} \mid \Theta = i) \cdot P(B(1000, \frac{4}{10})) = i}.
\end{align*}
Pravimo ji aposteriorna porazdelitev.


\section{Proučevani slučajni vektor (vzročni) parametrični model}

Naj bo $X = (X_1, X_2 \dots X_n) \in \R^n$ preučevani slučajni vektor.
Pogosto so neodvisni in enako porazdeljeni (NEP) realizacija danega slučajnega eksperimenta.
S pomočjo statistike lahko \sn{ocenjujemo} porazdelitev slučajnega vektorja $X$.
Zanjo privzamemo, da pripada nekemu modelu, t.j. neki množici dopustnih rešitev.
Privzamemo, da je ta množica parametrizirana s parametričnim prostorom $\Theta \subset \R^r$.
Tu si mislimo, da parameter $\theta \in \Theta$ dobimo kot realizacijo slučajnega vektorja (s.v.) $\Theta$
z vrednostmi v $\Theta$ (večinoma $r \geq 2$).
Porazelitvi s.v. $X_i$ pogojno na $\Theta = \theta$ pravimo vzorčna porazdelitev.
Privzeli bomo, da imamo gostote $f(x \mid \theta)$ ali verjetnostne funkcije
\begin{equation*}
  P(X = x \mid \theta) = f(x \mid \theta),
\end{equation*}
torej da velja
\begin{equation*}
  P(X \in B \mid \Theta = \theta) = \int_B f(x \mid \theta) d\nu (x)
\end{equation*}
(v Lebesgueovi meri) ali
\begin{equation*}
  P(X \in B \mid \Theta = \theta) = \sum_{x \in B} f(x \mid \theta).
\end{equation*}
Modelu pogojnih porazdelitev $(X \mid \Theta = \theta)$ pravino vrorčni model.


\section{Apriorna in \sn{robna} porazdelitev}

Porazdelitvi fiktivnega slučajnega vektorja $\Theta$ pravimo apriorna porazdelitev,
brezpogojni (robni) porazdelitvni slučajnega vektorja $X$ pa pravimo \sn{robna} porazdelitev \\
(*) v resnici sta obe porazdelitvi robni porazdelitvi družne porazdelitve vektorja $(X, \Theta)$ z vrednostmi v $\R^{n+r}$.
\begin{exmp}
  Ocenjujemo Bernoullijevo porazdelitev. Predhodno vedenje je podano z apriorno prazdelitvijo na $(0,1)$;
  mislimo si, da je $p$ realizacija slučajne spremenljivke (s.s.) $\Pi$ z vrednostmi v $(0,1)$. Možnosti:
  \begin{itemize}
    \item nimamo apriornega mnenja o (dejanskem) $p$: tedaj bi (morda) vzeli zvezno porazdelitev z gostoto enakomerna porazdelitve,
      % skica
    \item smo \sn{zelo} prepričani, da je (dejanski) $p \approx \frac{1}{2}$. %Tedaj bi imeli
      % skica
  \end{itemize}
\end{exmp}
Recimo, da je $f(p)$ gostota apriorne porazdelitve. Tedaj so apriorne verjetnosti
\begin{equation*}
  P(\Pi \in (a,b)) = \int_{a}^{b} f(p) dp
\end{equation*}
in apriorna pričakovana vrednost
\begin{equation*}
  E(\Pi) = \int_{0}^{1} p f(p) dp. 
\end{equation*}
Pripomnimo, da pri $\Pi \sim U(0,1)$ dobimo $E(U(0,1)) = \frac{1}{2}$. \\
Privzemimo, da smo \sn{vzorčili} $p$, potem pa \sn{neodvisno} $n$-krat vržemo $p$-kovanec ($P($cifra$=p)$),
gre za slučajne spremenljivke $X_1, X_2 \dots X_n$, za katere je $(X_i \mid \Pi = p) \sim Bernoulli(p)$
in so $X_1 \dots X_n$ neodvisne pogojno na $p$.
To ne pomeni, da do $X_1 \dots X_n$ brezpogojno neodvisne. \\
Za $i \neq j$ je
\begin{align*}
  P(X_i = 1 \land X_j = 1) &= \int_{0}^{1} P(X_i = 1 \land X_j = 1 \mid \Pi = p) f(p) dp = \\
  &\stackrel{\text{pogojno neodvisne}}{=}  \int_{0}^{1} P(X_i = 1 \mid \Pi = p) P(X_j = 1 \mid \Pi = p) f(p) dp = \\
  &= \int_{0}^{1} p^2 f(p) dp = \\
  &= E(\Pi^2).
\end{align*}
Ker je $P(X_i = 1) = \int_{0}^{1} P(X_i = 1 \mid p) f(p) dp = \int_{0}^{1} p f(p) dp = E(\Pi)$, je
\begin{equation*}
  Cov(X_i, X_j) = E(\Pi^2) - E(\Pi)^2 = D(\Pi)
\end{equation*}
za $i \neq j$, torej so $X_i$ brezpogojno neodvisne $\iff$ $\Pi =$ konstantna (slučajna spremenljivka). \\
Tvorimo $X = X_1 + \dots + X_n \in \{0, 1 \dots n\}$.
To je \sn{preučevana} slučajna spremenljivka.
Velja $(X\mid \Pi = p) \sim B(n,p)$.
To je vzorčna porazdelitev; vzročni model je parametriziran s prostorom parametrov $(0,1) = \Theta$.
Robna porazdelitev je podana z verjetnostmi
\begin{align*}
  P(X = k) &= \int_{0}^{1} P(X = k \mid p) f(p) dp = \\
  &= \int_{0}^{1} \binom{n}{k} p^k (1-p)^{n-k} f(p) dp.
\end{align*}
Recimo, da \sn{opazimo} $X = k$. Aposteriorna porazdelitev (osveženo vedenje o $p$) je sestavljeno iz verjetnosti
\begin{align*}
  P(X \in (a,b) \mid X = k) &= \frac{P(X = k \land \Pi \in (a,b))}{P(X = k)} = \\
  &= \frac{\int_{0}^{1} P(X = k \land \Pi \in (a,b) \mid \Pi = p) f(p) dp}{P(X = k)} = \\
  &= \int_{a}^{b} \frac{P(X = k \mid \Pi = p)}{P(X = k)} f(p) dp.
\end{align*}
Opazimo, da ima aposteriorna porazdelitev $(\Pi \mid X = k)$ gostoto
\begin{equation*}
  f_{(\pi \mid X)}(p \mid k) = \frac{P(X = k \mid p) f(p)}{P(X = k)}.
\end{equation*}
Zgornji formuli pravimo Bayesova formula. \\
Za številsko oceno za $p$ bi lahko vzeli pričakovano vrednost aposteriorne porazdelitve
\begin{equation*}
  \hat{p} = E(\Pi \mid X = k) = \int_{0}^{1} p \cdot f(p \mid k) dp.
\end{equation*}
Pravimo ji aposteriorna pričakovana vrednost. \\
Posebej priročna družina apriornih porazdelitev (za binomske vzorčne porazdelitve) je t.i.
$Beta = \{Beta(a,b) \mid a,b \in (0, \infty)\}$
\begin{equation*}
  f_{Beta(a,b)}(p) = \frac{1}{B(a,b)} p^{a-1} (1-p)^{b-1} 1_{(0,1)}(p)
\end{equation*}
(tu je $B(a,b) = \int_{0}^{1} p^{a-1} (1-p)^{b-1} dp$). \\
\begin{align*}
  &E(Beta(a,b)) = \frac{a}{a+b} \\
  &D(Beta(a,b)) = \frac{ab}{(a+b)^2 (a+b+1)}.
\end{align*}
$D(Beta(a,b))$ predstavlja \sn{težo} apriornega prepričanja; večji - manj sigurni smo.
% skica
\begin{equation*}
  E(Beta(a,b)) = 0.7.
\end{equation*}
Aposteriorna porazdelitev ima gostoto (če je $f(p) = f_{Beta(a,b)}(p)$)
\begin{align*}
  f(p \mid k) &= \frac{\binom{n}{k} p^k (1-p)^{n-k} \cdot \frac{1}{B(a,b)} p^{a-1} (1-p)^{b-1}}{P(X = k)} = \\
  &= \text{konst.} \cdot p^{a+k-1} (1-p)^{b+n-k-1}.
\end{align*}
Vidimo, da je $(\Pi \mid X = k) \sim Beta(a+k, b+n-k)$. \\
Aposteriorna pričakovana vrednost (p.v.) je
\begin{align*}
  \frac{a+k}{a+b+n} &= \frac{(a+b)\frac{a}{a+b} + n \frac{k}{n}}{a+b+n} = \\
  &= \frac{a+b}{a+b+n} \cdot \frac{a}{a+b} + \frac{n}{a+b+n} \cdot \frac{k}{n}.
\end{align*}
Tukaj je
\begin{itemize}
  \item $\frac{a}{a+b}$ apriorna ocena,
  \item $\frac{k}{n}$ vzorčna ocena in
  \item $\frac{a+b}{a+b+n}$ in $\frac{n}{a+b+b}$ faktorja pri konveksni kombinaciji obeh ocen.
\end{itemize}
Vzorec velik $\to$ prevlada mnenje vzorca.


% 2. predavanje: 3.10.

\section{Disperzija aposteriornih porazdelitev}

Gre pravzaprav za disperzijo pogojnih porazdelitev.
Naj bosta $X: \Omega \to \R^m$ in $Y: \Omega \to \R^n$ in naj ima $(X,Y)$ gostoto $f_{(X,Y)}$ glede na $\mu \times \nu$
Sledita gostoti \\
$f_X(x) = \int f_{(X,Y)}(x,y) d\nu (y)$ za $X$ glede na $\mu$ in \\
$f_Y(y) = \int f_{(X,Y)}(x,y) d\mu (x)$ za $Y$ glede na $\nu$.
Dalje definiramo pogojni porazdelitvi $(Y \mid X = x)$ in $(X \mid Y = y)$ preko gostot
\begin{equation*}
  f_{(Y \mid X)}(y \mid x) = \frac{f_{(X,Y)}(X,Y)}{f_X(x)}
\end{equation*}
glede na $\nu$: gostota v $X \to \mu$ in simetrično za $f_{(X \mid Y)}(x \mid y)$. \\
$P(Y \in B) \mid X = x = \int_B f_{(Y \mid X)}(y \mid x) d\nu (y)$ - porazdelitev, opremljena z gostoto.
\begin{defn}
  \begin{equation*}
    E(Y \mid X = x) = \int y f_{(Y \mid X)}(y \mid x) d\nu (y).
  \end{equation*}
  $y$ lahko zamenjamo s $h(y)$. \\
  Pišemo $E(Y \mid X = x) = u(X)$ - $h$ je identiteta.
\end{defn}
\begin{defn}
  \begin{equation*}
    E(Y \mid X) = u(X): \Omega \to \R^n.
  \end{equation*}
  Slučajni vektor $\to$ pogojna pričakovana vrednost, \\
  oz.
  \begin{equation*}
    E(Y \mid X)(\omega) = u(X(\omega)) = E(Y \mid X = X(\omega)).
  \end{equation*}
  $E(Y \mid X)(\omega)$: funkcija na $X$, kompozitum. \\
  $X(\omega)$: vrednost.
\end{defn}
\begin{defn}
  Pogojno varianco slučajnega vektorja $Y$, pogojno na $X = x$ definiramo kot varianco pogojne porazdelitve $(Y \mid X = x)$, t.j.
  \begin{equation*}
    E((Y - u(X))(Y - u(X))^T \mid X = x) =: Var(Y \mid X = x).
  \end{equation*}
  Ker je $E$ aditivna, velja
  \begin{equation*}
    E((Y - u(X))(Y - u(X))^T \mid X = x) = E(Y Y^T \mid X = x) - u(X) u(X)^T =: v(X).
  \end{equation*}
  $v(X)$ je $n \times n$ matrika.
\end{defn}
\begin{defn}
  Pogojna varianca slučajnega vektorja $Y$ pogojno na slučajni vektor $X$ je
  \begin{equation*}
    Var(Y \mid X) = v(X).
  \end{equation*}
\end{defn}




\clearpage
\phantomsection

\addcontentsline{toc}{chapter}{Literatura}
\bibliography{../bibtex/literatura}
\bibliographystyle{plainnat}


\clearpage
\phantomsection

\chapter*{Dodatki}
\addcontentsline{toc}{chapter}{Dodatki}




\end{document}
